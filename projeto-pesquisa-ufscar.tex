%% Este arquivo foi modificado de:
%% modeloABNT2, v1.0 athila
%% Copyright 2013 by Athila e Monaro
%% Desenvolvido para EESC-USP

\documentclass[projeto]{ufscar}

% ---
% PACOTES
% ---

% ---
% Pacotes fundamentais 
% ---
\usepackage{cmap}				% Mapear caracteres especiais no PDF
\usepackage{lmodern}				% Usa a fonte Latin Modern
\usepackage{makeidx}            	% Cria o indice
\usepackage{hyperref}  			% Controla a formação do índice
\usepackage{lastpage}			% Usado pela Ficha catalográfica
\usepackage{indentfirst}			% Indenta o primeiro parágrafo de cada seção.
\usepackage{nomencl} 			% Lista de simbolos
\usepackage{graphicx}			% Inclusão de gráficos
% ---

% ---
% Pacotes adicionais
% ---
\usepackage{lipsum}				       % para geração de dummy text
\usepackage[printonlyused]{acronym}
\usepackage[table]{xcolor}
% ---


% ---
% Informações de dados para CAPA
% ---
%
% Título:
%	1. Título em português
%	2. Título em inglês
\titulo{Título em português}{Título em inglês}
%
% Autor:
%	1. Nome completo do autor
%	2. Formato de nome para bibliografia
\autor{Nome completo do autor}{Sobrenome, Nome}
%
% Cidade
\local{São Carlos}
% Ano de defesa
\data{2020}

% Universidade
\universidade{Universidade XXXXX}{XXXXX}
% \universidade{Universidade Federal de São Carlos}{UFSCar}

% Centro
\centro{Centro de XXXXX}{XXXX}
% \centro{Centro de Ciências Exatas e de Tecnologia}{CCET}

% Depatamento
\departamento{Departamento de XXXX}{XX}
%\departamento{Departamento de Computação}{DC}
%\departamento{Departamento de Engenharia Elétrica}{DEE}

% Programa de Pós-Graduação
\programa{Programa de Pós-Graduação em XXXXXX}{PPGXX}
% \programa{Programa de Pós-Graduação em Ciência da Computação}{PPGCC}
%\programa{Programa de Pós-Graduação em Engenharia Elétrica}{PPGEE}

% Curso
\curso{XXXXXX}
%\curso{Ciência da Computação}
%\curso{Engenharia Elétrica}


% Área de concentração da pesquisa
\areaconcentracao{XXXXXX}
% \areaconcentracao{Metodologias e Técnicas de Computação}
% \areaconcentracao{Sistemas Elétricos e Eletrônicos}

% Nome do orientador
\orientador{Nome completo do orientador}
% Nome do coorientador
\coorientador{Nome completo do coorientador}
% ---

% ---
% compila o indice
% ---
\makeindex
% ---

% ---
% Compila a lista de abreviaturas e siglas
% ---
\makenomenclature
% ---

% ----
% Início do documento
% ----

\begin{document}

% ----------------------------------------------------------
% ELEMENTOS PRÉ-TEXTUAIS
% ----------------------------------------------------------
\pretextual

% ---
% Insere Capa, Folha de rosto.
% ---
\maketitle

% ---
% inserir o sumario
% ---
\sumario
% ---

% ----------------------------------------------------------
% ELEMENTOS TEXTUAIS
% ----------------------------------------------------------
\mainmatter

% ----------------------------------------------------------
% Informações Gerais
% ----------------------------------------------------------
\chapter{Informações Gerais}
%\addcontentsline{toc}{chapter}{Informações Gerais}

\section{Título do Projeto}

Título do projeto.


\section{Palavra-Chave}

Palavra 1, Palavra 2.


\section{Grupo de Pesquisa de Vinculação}

Pesquisa vinculada ao grupo de Robótica / Mecatrônica (Grupo de Pesquisa Certificado pela UFSCar e pelo CNPq) (http://www.robotica.ufscar.br).

\section{Duração Total do Projeto}
\begin{itemize}
    \item Número de Meses: 24
    \item Período Abrangido: 03/2019 à 03/2020
\end{itemize}
\section{Responsáveis pela Formulação do Projeto}
\begin{itemize}
    \item Nome do Proponente: Aluno
    \item Nome do Professor: Professor
\end{itemize}

\section{Local de Execução do Projeto}

\indent Universidade Federal de São Carlos –- UFSCar\\ 
\indent Centro de Ciências Exatas e Tecnologia –- CCET\\


\indent Rodovia Washington Luís, km 235 - SP-310 \\
\indent São Carlos, SP, CEP 13565-905 \\
\indent Fone: (16) 3351-8607 

\section{Resumo do Projeto de Pesquisa}

\section{Abstract}

% ----------------------------------------------------------
% Descrição do Projeto de Pesquisa
% ----------------------------------------------------------
\chapter{Descrição do Projeto de Pesquisa}
% \addcontentsline{toc}{chapter}{Descrição do Projeto de Pesquisa}

\section{Introdução}

\cite{ibge1993}

\section{Justificativas e Objetivos do Projeto de Pesquisa}

\section{Aspectos Metodológicos}

\section{Plano de Atividades e Cronograma}

% ---
\bookmarksetup{startatroot}% 
% ---
% Considerações Finais
% ---
\section{Considerações Finais}
% \addcontentsline{toc}{chapter}{Considerações Finais}


% ----------------------------------------------------------
% Referências bibliográficas
% ----------------------------------------------------------

\bibliography{refs/abntex2-modelo-references}


% ----------------------------------------------------------
% Apêndices
% ----------------------------------------------------------
% ---
% Inicia os apêndices
% ---
%\begin{apendicesenv}
% Imprime uma página indicando o início dos apêndices
% \partapendices
% ----------------------------------------------------------
% Incluir Apêndice
% ----------------------------------------------------------
%\chapter{Quisque libero justo}
%\lipsum[1-5]

%\end{apendicesenv}
% ---

% ----------------------------------------------------------
% Anexos
% ----------------------------------------------------------
% ---
% Inicia os anexos
% ---
%\begin{anexosenv}
% Imprime uma página indicando o início dos anexos
% \partanexos
% ---
% Incluir Anexo
% ---
%\chapter{Morbi ultrices rutrum lorem.}
%\lipsum[1-25]
%\section{Test}
%\lipsum[1-20]


%\end{anexosenv}


\end{document}


